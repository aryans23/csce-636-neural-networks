\documentclass[letterpaper]{article}

\usepackage[english]{babel}
\usepackage[utf8]{inputenc}
\usepackage{amsmath}
\usepackage{graphicx}
\usepackage[colorinlistoftodos]{todonotes}
\usepackage[]{algorithm2e}
\usepackage{listings}
\usepackage{hyperref}
\usepackage{float}
\usepackage{afterpage}

\title{CSCE 636: Homework 1}

\author{Aryan Sharma\\UIN: 326006767} % The article author(s) - author affiliations need to be specified in the AUTHOR AFFILIATIONS block

\date{} % An optional date to appear under the author(s)

\begin{document}
	
\maketitle

\noindent {\Large Question 1}\\

\noindent \textbf{1a.} The function train\_valid\_split, splits the dataset into training and validation sets. The training dataset is the actual dataset that we use to train the model. However, if we use the entire training dataset, we can overfit and may not generalize well to the test data. The validation dataset is thus used to tune the hyperparameters. By testing on a set of validation examples that the models were not trained on, we obtain a better estimate of each hypothesis hi’s true gene\\

\noindent \textbf{1b.} We can use the full dataset to produce our final model as the more data we use the more likely it is to generalise well but we should also make sure that we obtain an unbiased performance estimate via nested cross-validation and potentially consider penalising the cross-validation statistic to further avoid over-fitting in model selection.\\

\noindent \textbf{1c.} Implemented in code\\

\noindent \textbf{1d.} This is a bias addition in the Weight matrix.\\

\noindent \textbf{1e.} Implemented in code\\

\noindent \textbf{1f.} Shown in Fig. 1\\\\

\noindent {\Large Question 2}\\

\noindent \textbf{1a.} Implemented in code\\

\noindent \textbf{1b.} Implemented in code\\

\noindent \textbf{1c.} Shown in Fig. 2\\

\noindent \textbf{1d.} Implemented in code. The accuracy is -----.\\\\

% \begin{figure} 
% 	\centering
% 	\includegraphics[width=1\textwidth]{.png}
% 	\caption{\label{fig:data}Number of Disk I/Os vs Relation Size}
% \end{figure}



%\begin{table}
%	\centering
%	\begin{tabular}{|c|c|c|c|}
%		\hline
%		Table1 Size & Table2 Size & Disk I/Os & Time(ms) \\\hline
%		20 & 20 & 3593 & 266906\\
%		30 & 30 & 4212 & 387113\\
%		50 & 50 & 7457 & 552559\\
%		70 & 70 & 16195 & 1.25760e+06\\
%		100 & 100 & 22418 & 1.66148e+06\\ 
%		200 & 200 & 48057 & 3.55844e+06\\ \hline
%	\end{tabular}
%	\caption{\label{tab:widgets}Comparison of Total Time taken and Number of Disk I/Os as Relation size increases post optimization}
%\end{table}

%\begin{figure} 
%	\centering
%	\includegraphics[width=1\textwidth]{dio.png}
%	\caption{\label{fig:data}Number of Disk I/Os vs Relation Size}
%\end{figure}
%
%\begin{figure} 
%	\centering
%	\includegraphics[width=1\textwidth]{time.png}
%	\caption{\label{fig:data}Time elapsed vs Relation Size}
%\end{figure}

\end{document}